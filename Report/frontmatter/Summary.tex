%!TEX root = ../Report.tex
\chapter{Resumo}
O AMazING  (Advanced Mobile wIreless Network playGround) é uma plataforma de testes que tem como objetivo a capacidade de agendar e executar experiências de networking. A rede AMazING é composta por 24 nós fixos com capacidades wireless e um nó móvel (denominado de robô). \newline\\
Através desta plataforma, de uma forma simples e intuitiva, podemos realizar experiências que podem variar entre o teste de novos protocolos a testes de interferência em redes wireless. Adicionalmente o robô, que possui um nó, está posicionado num carril o que o permite simular os movimentos de um utilizador à medida que se  movimenta em relação aos 24 nós fixos. \newline\\
Com principal objetivo deste projeto foi estabelecido que este seria o desenvolvimento de uma interface que facilite a interação com os nós acima mencionados e que permita, também, aos utilizadores fazer o agendamento e gestão das suas experiências na plataforma, bem como analisar e visualizar os resultados obtidos em cada uma das suas experiências. É importante realçar que o trabalho a ser realizado com o robô ficou, após diálogo com os orientadores, para segundo plano. Isto devido a uma reestruturação eletrónica no âmbito de trabalhos a ser desenvolvidos em paralelo. Tais motivos tornaram o robô indisponível na realização deste projeto.\newline\\
Neste documento é apresentado o estudo efetuado relativamente a esta plataforma de testes bem como o seu processo de implementação. De seguida, é descrita a implementação da plataforma, a sua arquitetura e o seu modelo de dados.


\chapter{Palavras-Chave}
\begin{acronym}[Keys]
\acro{Experiência}{Testes realizados sobre os nós existentes na rede}
\acro{APU: Advanced Processing Unit}{Hardware similar a um router utilizado para realizar as experiências de rede}
\acro{SBC: Single Board Computer}{Computador construído numa única placa de circuitos. Possui elementos como um microprocessador(es), memória e I/O. }
\acro{Nó}{APU configurada}
\end{acronym}

\chapter{Agradecimentos}
Gostaríamos de agradecer aos supervisores de projeto, o Professor Flávio Meneses e ao Professor Daniel Corujo pela disponibilidade que demonstraram desde o primeiro momento para nos  ajudar a chegar aos objetivos estabelecidos e a clarificar dúvidas que foram surgindo ao longo das várias fases do projeto.
