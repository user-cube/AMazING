%!TEX root = ../Report.tex
\chapter{Requisitos e Arquitetura do Sistema}
\label{chp:requirements}
\section{Requisitos do Sistema}
Esta secção apresenta as especificações gerais dos requisitos do sistema. As seguintes subseções descrevem, em primeiro lugar, o processo de levantamento de requisitos e de seguida o contexto, os atores e os respetivos casos de uso. Por fim, são especificados os requisitos não funcionais do sistema, bem como as dependências do sistema e suposições.\newline
O levantamento de requisitos foi realizado na primeira fase de desenvolvimento do projeto,  na fase inicial de desenvolvimento. Foi realizada uma reunião com os Professore Flávio Meneses e o Professor Daniel Corujo na qual foram expostos os principais objetivos da plataforma.\newline
Desse modo, foram tomadas as seguintes decisões:\newline
\begin{itemize}
    \item Uma plataforma web será o foco principal do projeto sendo que a possibilidade de trabalhar com o nó móvel (também denominado de robô) foi movida para trabalho futuro devido ao facto do mesmo estar a ser alvo de reestruturações mecânicas e eletrónicas;
    \item A utilização da plataforma está restrita para utilizadores autenticados.
        \SubItem{Numa primeira fase serão criados, por um administrador do sistema, utilizadores próprios da plataforma (com um sistema de login com utilizador e palavra passe);}
        \SubItem{Numa fase posterior perspectiva-se a integração com as credenciais de colaboradores do IT através do LDAP.}
    \item Permitir a reconfiguração da rede de nós através da plataforma assim como a visualização dos dados de cada nó e os resultados das variadas experiências;
\end{itemize}

\subsection{Contexto}
Redes sem fios e, mais em particular redes móveis, têm sido um assunto bastante discutido nos dias que correm. Trabalhos de investigação e de inovação têm-se deparado com dificuldades na realização de testes ou experiências em ambiente real. Tal pode acontecer devido ao preço do material, condições ambientais ou difícil reprodução das condições ou ambientes sem fios. Para solucionar este problema surge a plataforma AMazING, uma plataforma de testes wireless com uma interface apelativa e fácil de utilizar. \newline
A plataforma foi desenvolvida de modo a facilitar a configuração dos nós necessários para cada experiência. O utilizador poderá agendar e realizar a sua experiência num ambiente o mais real possível. Importante referir que a execução de vários testes em simultâneo seria possível desde que fossem utilizados nós distintos. Por questões de simplicidade, sempre que um utilizador pretender reservar uma experiência, todos os nós ficam reservados para ele durante um determinado espaço de tempo. (é inexequível).
No menu principal está disposta a grelha de nós e as várias operações possíveis estão divididas em secções na barra de navegação e cada secção pode ter subseções que permitem realizar funcionalidades mais específicas.

\subsection{Atores}
A desenvolvimento da interface gráfica foi levado a cabo tendo em conta que a plataforma será utilizada por pessoas que à partida são experientes na sua área, porém mantendo a interface simples de modo a ser de uso fácil. De seguida, enumeramos uma classificação dos atores que irão utilizar a plataforma:
\begin{itemize}
    \item \textbf{Utilizador - Begginer}  – O João é estudante de Mestrado e está a realizar a sua tese no Instituto de Telecomunicações. Encontra-se num projeto de investigação que envolve testes de comunicação (com e sem fios) e desenvolvimento de novos protocolos de redes.\newline 
    O João precisa de realizar os seus testes num ambiente o mais real possível evitando simuladores virtuais. \newline
    O João é um utilizador iniciante, (nível 1) e só lhe é permitido utilizar os templates ao utilizar o sistema. e carregar ficheiros de execução para as APUs.
    \item \textbf{Utilizador - Advanced}  - O João recebe acesso de utilizador avançado (nível 2) o que lhe permite ter acesso direto ao terminal das APUs durante a realização de sua experiência. Podendo assim realizar a instalação de drivers, implementação de protocolos e outros serviços necessários para a realização da experiência
    \item \textbf{Administrador} - O Manuel é professor e Doutorado em Redes e Telecomunicações. É um dos responsáveis pelo Instituto de Telecomunicações e tem experiência em vários projetos relacionados com o desenvolvimento e implementação de novos protocolos de redes e telecomunicações.\newline
    O Manuel precisa de garantir o funcionamento e o acesso aos recursos disponibilizados no IT. 
\end{itemize}

\subsection{Casos de Uso}
Nas imagens seguintes, apresentamos os modelos de casos de uso(CaU) da solução. 
\subsubsection{Utilizador/Tester}
    \begin{figure}[!ht]
        \centering
        \includegraphics[height=0.4\textheight]{images/CaU1.png}
        \caption{Casos de uso do utilizador/Tester}
        \label{fig:usage}
    \end{figure}
    Os casos de usos consoante a sua prioridade podem sintetizados de acordo com a seguinte tabela:
    \begin{table}[ht]
        \centering
            \begin{tabular}{p{.25\textwidth}p{.50\textwidth}p{.15\textwidth}}
                \hline
                \textbf{CaU} &	\textbf{Descrição} &	\textbf{Prioridade} \\ 
                \hline
                Configurar Nó & Permite ao utilizador configurar um dado nó da rede como preferir. & Alta \\
                \hline
                Agendar/Fazer teste & Permite ao utilizador agendar a sua experiência e consequentemente efetuá-la no seu tempo alocado. & Alta \\
                \hline
                Verificar nó & Permite ao utilizador ver a informação de um nó (o seu estado, configuração, etc) & Alta \\
                \hline
                Recolher Dados & Permite ao utilizador recolher os dados provenientes da sua experiência. & Alta \\
                \hline
                Solicitar acesso de nível 2 & Permite que utilizador solicite a um administrador uma elevação dos seus privilégios na plataforma. & Alta \\
                \hline
            \end{tabular}
        \caption{Casos de uso do utilizador/Tester}
        \label{myTable}
    \end{table}

\subsubsection{Administrador}
\begin{figure}[!ht]
    \centering
    \includegraphics[height=0.4\textheight]{images/CaU2.png}
    \caption{Casos de uso do Administrador}
    \label{fig:usage}
\end{figure}
Os casos de usos consoante a sua prioridade podem sintetizados de acordo com a seguinte tabela:
\begin{table}[ht]
    \centering
        \begin{tabular}{p{.25\textwidth}p{.50\textwidth}p{.15\textwidth}}
            \hline
            \textbf{CaU} &	\textbf{Descrição} &	\textbf{Prioridade} \\ 
            \hline
            Acesso a estatísticas & Permite ao administrador aceder às estatísticas da plataforma. & Alta \\
            \hline
            Gerir Utilizadores & Permite ao administrador gerir os utilizadores da plataforma assim como elevar o seu nível. & Alta \\
            \hline
            Adicionar/alterar Configurações padrões & Permite ao administrador adicionar/alterar configurações padrão dos nós & Alta \\
            \hline
            Verificar estado atual & Permite ao administrador verificar o estado atual da rede AMazING e dos seus nós. & Alta \\
            \hline
            Gerir Calendário & Permite ao administrador ver e gerir o calendário e os testes agendados. & Alta \\
            \hline
        \end{tabular}
    \caption{Casos de uso do Administrador}
    \label{myTable}
\end{table}

\newpage

\subsection{Requisitos não funcionais}
Abaixo, apresentamos os requisitos não funcionais do sistema.
\begin{itemize}
    \item \textbf{Usabilidade: } dado o principal propósito da aplicação, é necessário que esta seja simples de aprender e utilizar;
    \item \textbf{Autenticação:} A utilização da aplicação deve ser disponível apenas para utilizadores do IT 
        \SubItem{Numa primeira fase, adicionados ao sistema por um administrador;}
        \SubItem{Numa segunda fase, possuindo autenticação pelas credenciais de colaboradores do IT através do LDAP.}
    \item \textbf{Controlo:} O administrador do sistema deve ser capaz de configurar e realizar a manutenção do sistema;
    \item \textbf{Dados:} Os utilizadores do sistema deve conseguir aceder aos dados da experiência realizada.

\end{itemize}