%!TEX root = ../Report.tex
\chapter{Introdução}
\label{chp:introduction}
O desenvolvimento de plataformas que facilitem as operações de utilizadores de um dado sistema não é um conceito novo.  No entanto, cada um destes sistemas aborda uma tarefa ou comportamento que uma percentagem considerável da população adota enquanto que o nosso caso particular é um pouco mais “exclusivo”. E por esse mesmo motivo há uma amostra menor de plataformas que têm como objetivo auxiliar estes utilizadores.

\section{Contexto}
Este projeto foi desenvolvido no âmbito da unidade curricular de Projeto em Informática, inserido na Licenciatura em Engenharia Informática (LEI) da Universidade de Aveiro (UA). O foco do projeto é criação de uma interface que permita fazer o agendamento e gestão de experiências na plataforma de testes, sendo possível consultar os resultados após o término das mesmas. O presente trabalho foi desenvolvido durante o segundo semestre do ano letivo de 2019/2020.

\section{Motivação}
A plataforma AMazING constitui uma mais valia para qualquer pessoa que pretenda realizar testes sobre redes móveis dado que esta permite simular situações reais.\newline
A motivação para este projeto é o desenvolvimento uma plataforma que ajude essas mesmas pessoas a agendar e gerir as suas experiências e a obter os dados resultantes das mesmas, desse modo, facilitando todo o processo através de um GUI que lhes fornece tudo o que necessitam de forma fácil e intuitiva.

\section{Objetivos}
Este projeto teve como principal objetivo desenvolver uma plataforma que intuitiva e de fácil utilização que permita aos seus utilizadores realizar as suas experiências com um nível menor de esforço no que toca à preparação  e  visualização destas.\newline
Após uma reunião com os orientadores do projeto procedeu-se à definição de uma lista de objetivos na qual o desenvolvimento deste projeto se focou. Os objetivos definidos podem ser enunciados da seguinte forma:
\begin{itemize}
    \item Criação de uma plataforma web que agrega todo o trabalho já existente;
    \item Tornar a plataforma intuitiva e fácil de usar como foi mencionado anteriormente;
    \item Tornar a plataforma de testes reconfigurável de modo a que esta se adapte às necessidades de cada utilizador.
    \item Desenvolvimento de uma ferramenta que permite a visualização de informação de cada nó existente na infraestrutura bem como recolher estatísticas e informações importantes numa dada experiência;
    \item A plataforma deve ser capaz de gerir ambientes de multi-teste.
\end{itemize}

\section{Impactos do Covid-19}
Durante o desenvolvimento deste projeto houve a clara interferência causada pela situação do Covid-19 e como tal nós reunimos e delineamos não só os impactos mas também as soluções de modo a melhor nos prevenirmos e de modo a que o desenvolvimento não fosse afetado.
Deste modo consideramos que os maiores impactos seriam:
\begin{itemize}
    \item A não existência de reuniões presenciais o que dificulta a comunicação e interação do grupo;
    \item A impossibilidade de reuniões presenciais com os nosso orientadores, tornando todo o processo de transmissão de informação e esclarecimento de dúvidas mais complicado;
    \item A impossibilidade de aceder fisicamente ao IT o que impede as experiências com os equipamento lá presentes, nomeadamente o Switch onde estão conectados os nós, assim como acesso à rede privada de testes;
    \item Devido à impossibilidade de acesso à rede privada de testes foram disponibilizadas 5 APUs para nos possibilitar o avanço do trabalho sem a necessidade de deslocamento ao IT. Seria um para cada membro do grupo no entanto com a divisão de tarefas apenas dois membros do grupo ficaram com APUs - 3 e 2 para os 2 membros encarregues desta parte do projeto;
    \item Impacto da moral e motivação da equipa uma vez que não era possível reunirmo-nos presencialmente e ver os resultados da solução tão claramente.
\end{itemize}

\break